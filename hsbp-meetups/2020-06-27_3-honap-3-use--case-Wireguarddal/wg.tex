\documentclass[aspectratio=169]{beamer}
\beamertemplatenavigationsymbolsempty
\usepackage{tcolorbox}
\usepackage{t1enc}
\usepackage[magyar]{babel}
\usepackage[utf8]{inputenc}

%\setbeamertemplate{navigation symbols}{}%remove navigation symbols

\usepackage{fontspec}
%\setmainfont[Mapping=tex-text]{Bebas Neue}
\setmainfont[Mapping=tex-text]{Asap}
\let\sfdefault\rmdefault

\AtBeginSection[]

%\usetheme{Warsaw}
\usecolortheme{frigatebird}
%\usetheme{Szeged}

\tcbset{%
    noparskip,
    colback=noir!90, %background color of the box
    colframe=gray!40, %color of frame and title background
    coltext=blanc, %color of body text
    coltitle=black, %color of title text 
    fonttitle=\bfseries,
    alerted/.style={coltitle=red, 
                     colframe=gray!40},
    example/.style={coltitle=black, 
					 colframe=myOrange!100,},
%                     colback=green!5},
    }

\newfontfamily\bebasfont{Bebas Neue}

%Information to be included in the title page:
%\title{\bebasfont 3 hónap:\ 3 use-case WireGuard-dal}
\title{3\,hónap: 3\,use\,case WireGuard-dal}
\author{Szabó Endre}
\institute[HSBP Meetup]{Hackerspace Budapest Online Meetup}
\date[\today]{\today}

\begin{document}

{
%\setbeamercolor{background canvas}{bg=violet}
\frame{\titlepage}
}

\section{Áttekintés}
\subsection{terjedelem}

\begin{frame}{Az előadás terjedelme}

	Mi nem célja az előadásnak?
		\pause
		\begin{itemize}
			\item Ez nem WireGuard endorsement. YMMV.
		\end{itemize}

		\pause

	Mi a célja az előadásnak?

		\pause

		\begin{itemize}
			\item Évek alatt kialakult best-practice-ok kivonatos bemutatása.
			\item Otthoni VPN környezetem bemutatása.
		\end{itemize}

	%	\includegraphics[scale=0.2]{scb-magyarazos.png}

%\bebasfont teszt 123

\end{frame}

\subsection{halozat}

\begin{frame}
	\frametitle{IP-címzés általánosságban}

	\textbf {Site:} minden, fizikailag elkülönülő telephelyen lévő eszközök összessége. Ez lehet:
	\pause
	\begin{itemize}
		\item valódi helyi hálózat (lakás),
	\pause
		\item VPS,
	\pause
		\item vagy egy-egy kiemeltebb road-warrior is.
	\end{itemize}
	\pause

	Általános jellemzők:
	\pause

	\begin{itemize}
		\item	AMPRNet\texttrademark\,\,44.128.0.0/16 teszt célú hálózatát lenyúltam,
	\pause
\item minden site-nak van egy ``$x<64$'' azonosítószáma (\texttt {site\_id}),
	\pause
\item	(ebből kalkulálódik egy /22 méretű IP subnet minden site-nak),
	\pause
\item és egy ENSZ LOCODE alapú azonosító stringje (\texttt {site\_code}),
	\pause
	\item	(ebből származtatódnak a hostnevek prefixei).
	\end{itemize}

\end{frame}

\section{1. Usecase}
\begin{frame}[plain]
	\fontsize{50}{60} \bebasfont {
		1. use case:\,\,\,\\
		Site to site VPN\,\,\,
		}
\end{frame}
\subsection{Site to Site VPN}

\begin{frame}{Site to Site VPN jellemzők}
	Jellemzők:
	\begin{itemize}
		\item Full mesh VPN
			\pause
		\item Automatikus deploy Ansible-lel
			\pause
		\item Teljes 0.0.0.0/0 hirdetve minden végponton
			\pause
		\item SNAT Internet felé minden, tunnelből érkező forgalomra
	\end{itemize}
			\pause
	Hátrány:
	\begin{itemize}
			\pause
		\item WireGuard crypto-routing és DynDNS miatt minden site felé egy dedikált tunnel interface jelenléte szükséges.
	\end{itemize}
\end{frame}

\begin{frame}{Site to Site VPN route optimalizálás}
	\pause
	Nemzetközi forgalom optimalizálásra is lehetőség van.

	A probléma:
	\begin{itemize}
		\item Bécsi UPC és DIGI között a forgalom: Bécs - Frankfurt - Bukarest - Budapest
			\pause
		\item 38 msec RTT.
			\pause
	\end{itemize}
	A megoldás:
	\begin{itemize}
		\item Forgalom elirányítása kevesebb hop-on keresztül, BGP nélkül.
			\pause
		\item Egyik budapesti VPS-en terminálás nélküli, same-interface tükrözés.
			\pause
		\item Natívan, kernelspace-ben, netfilter DNAT+SNAT kombóval.
			\pause
		\item A forgalom útja így: Bécs - Budapest (- Budapest)
			\pause
		\item Eredmény: 18 msec RTT.
	\end{itemize}
	Hátrány:
	\begin{itemize}
			\pause
		\item Dinamikus című végpontok között initiatort kell kinevezni.
			\pause
		\item SPoF bevezetése. :)
	\end{itemize}
\end{frame}

\section{2. Usecase}
\begin{frame}[plain]
	\fontsize{50}{60} \bebasfont {
		2. use case:\,\,\,\\
		Road warrior VPN\,\,\,
		}
\end{frame}
\subsection{Road warrior VPN}

\begin{frame}{Road warrior VPN jellemzők}
	Jellemzők:
	\begin{itemize}
		\item Ansible-lel segített
			\pause
		\item Teljes 0.0.0.0/0 hirdetve minden koncentrátoron
			\pause
		\item Jobb kliensek párhuzamosan több tunnelt is tudnak használni
			\pause
		\item Kliens oldalon network namespacing kihasználása
			\pause
		\item Szerver oldalon proxy ARP lehetősége
			\pause
	\end{itemize}
			\pause
	Namespace használat macerái:
	\begin{itemize}
		\item Nincs network manager ami támogatná ezt a setup-ot.
			\pause
		\item Még a \texttt {wg-quick} sem támogatja, de patch-et küldtem.
			\pause
		\item Default namespace helyi hálózatra macerásan tud szolgáltatni.
	\end{itemize}
\end{frame}

\section{3. Usecase}
\begin{frame}[plain]
	\fontsize{50}{60} \bebasfont {
		3. use case:\,\,\,\\
		Magyar fix IP-cím\,\,\,
		}
\end{frame}
\subsection{Magyar IP-cím}

\begin{frame}{Magyar IP-cím}
	Szükség volt egy magyar IP-cím Bécsbe routolására.

	Helyi hálózati jellemzők:
	\begin{itemize}
		\item Két WiFi SSID itthon:
			\pause
		\item Egy natív, bécsi UPC kijáratú
			\pause
		\item Egy tunnelezett, budapesti IP-címmel rendelkező kijáratú
			\pause
		\item Snowflake network bridge
			\pause
		\item PVLAN és WiFi client isolation alapú \texttt{fwmark}
			\pause
		\item RPDB rule \texttt{fwmark} alapján
	\end{itemize}
	\pause
	Hátrány:
	\begin{itemize}
		\item Nincs.
	\end{itemize}
\end{frame}

\end{document}
